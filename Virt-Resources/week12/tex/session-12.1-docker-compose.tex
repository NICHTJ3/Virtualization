% Beamer slide template prepared by Tom Clark <tom.clark@op.ac.nz>
% Otago Polytechnic
% Dec 2012

\documentclass[10pt]{beamer}
\usetheme{Dunedin}
\usepackage{graphicx}
\usepackage{fancyvrb}

\newcommand\codeHighlight[1]{\textcolor[rgb]{1,0,0}{\textbf{#1}}}

\title{Docker Compose}

\author[I720]{Virtualisation}
\institute[Otago Polytechnic]{
  Otago Polytechnic \\
  Dunedin, New Zealand \\
}
\date{}
\begin{document}

%----------- titlepage ----------------------------------------------%
\begin{frame}[plain]
  \titlepage
\end{frame}

\begin{frame}
  \frametitle{Think about what we did last time}
  
  \begin{enumerate}
    \item Build \texttt{flaskapp} image
    \item Build \texttt{nginx} image
    \item Create \texttt{app} network
    \item Run a \texttt{flaskapp} container, attaching it to the \texttt{app} network
    \item Run an \texttt{nginx} container, attaching it to the \texttt{app} network and mapping its port 80 to host port 8080
  \end{enumerate}
\end{frame}

\begin{frame}
  \frametitle{This seems like something we might do more than once}
  
   How do we deal with a situation like this?
   
   \begin{enumerate}
     \item Write a shell script.
     \item Use (or create) a more general tool.
   \end{enumerate}
   
   A more general tool exists: \emph{Docker Compose}.

\end{frame}

\begin{frame}
  \frametitle{How would we redo the last lab with Docker Compose?}
   
   \begin{enumerate}
     \item Create a project directory, called \texttt{lab11}.
     \item Place the build contexts for our images inside this directory.
     \item Create a file, \texttt{docker-compose.yml} in the directory with appropriate directives.
   \end{enumerate}
   
   We can then build and run everything by executing the command \texttt{docker-compose up}.

\end{frame}

\begin{frame}[fragile]
    \frametitle{Our compose file}
    
    \begin{verbatim}
version: "2.0"
services:
    flaskapp:
        build: ./flaskapp
        image: user/flaskapp
        networks:
            - app
    nginx:
        image: nginx
        ports:
            - 8080:80
        networks:
            - app
        depends_on:
            - flaskapp
networks:
    app:
    \end{verbatim}
\end{frame}

\begin{frame}
  \frametitle{Compose file sections}
   \begin{itemize}
     \item \texttt{version:} May be 2.x or 3.x   
     \item \texttt{services:} Define your containers here
     \item \texttt{networks:} Declare your networks
     \item \texttt{volumes:} Define named volumes
   \end{itemize}
\end{frame}

\begin{frame}
  \frametitle{When is Docker Compose a good tool}
   
   \begin{itemize}
     \item Development environments    
     \item Automated builds, tests
     \item Small deployments
   \end{itemize}
   
 However, if you're going deploy services on a larger scale, you need more horsepower than is provided by Docker Compose.

\end{frame}


\begin{frame}
  \frametitle{More Information}
  
  See \url{https://docs.docker.com/compose/}
\end{frame}

\end{document}
