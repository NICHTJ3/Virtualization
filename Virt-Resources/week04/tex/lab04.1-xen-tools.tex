\documentclass{article}
\usepackage{graphicx}
\usepackage{wrapfig}
%\usepackage{inconsolata}
\usepackage{enumerate}
\usepackage{hyperref}
\usepackage{verbatim}
\usepackage[parfill]{parskip}
\usepackage[margin = 2.5cm]{geometry}

\usepackage[T1]{fontenc}


\begin{document}

\title{Lab 04.1: Xen-Tools\\ IN720 Virtualisation}
\date{}
\maketitle

\section*{Introduction}
A repeated\footnote{Some might even say tedious.} theme running through this part of the paper has been that Xen
by itself is too low-level for use in typical production settings. We need to either build or get additional management 
tools to carry out common tasks on our Xen hosts. There are several options available that span the range from some 
lightweight scripts to Enterprise\texttrademark management consoles.

In this lab we will consider something one the simple and lightweight end of the spectrum: Xen-Tools. These are basically just a set of Perl scripts and associated resources for creating and working with guest images. Xen-Tools is worth considering for two reasons: First, as a useful tool in it's own right; second as a model for how similar lightweight tools can be built.

\section{Install and use Xen-Tools}
The primary point of Xen-Tools is to easily create guest images running a variety of Linux operating
systems (Assorted versions of Debian, Ubuntu, and CentOS). Recall that we created our guest image 
by setting up a guest and manually installing the operating system on it. Xen-Tools provides a script,
\texttt{xen-create-image}, that automates this process. I also includes scripts to update, list, and delete 
guest images.

Since our Dom0 operating system is Ubuntu, installing Xen-Tools is utterly free of drama. Simply use 
\texttt{apt-get} to install the \texttt{xen-tools} package on our Dom0 system.

The main tool in the package is \texttt{xen-create-image}. Run the following command as an example:

\texttt{xen-create-image  --hostname=lab14img --lvm=<hostname>-vg --dist=trusty --memory=512M --dhcp}

Where we should substitute the name of your Dom0 host for \texttt{<hostname>}.

This will take a little time to run and produce some output. Some of the behaviour of the tool is governed by
a set of defaults, but we have to specify some items. In this example

\begin{description}
\item[\texttt{hostname}] (required) is the name the new guest image will have.
\item[\texttt{lvm}] is the name of the LVM volume group where the guest volumes will go.
\item[\texttt{trusty}] is the Linux distribution to install in the guest.
\item[\texttt{memory}] is exactly what you think it is.
\item[\texttt{dhcp}] specifies that the guest should be configured to use DHCP.
\end{description}

\section{Review the results}
The result of the command above should be placing a runnable guest on our Xen host. This means that 
three things have been put in place:

\begin{enumerate}
 \item Disk: LVM volumes have been created for the guest. Verify this with \texttt{lvs}.
 \item Configuration: A configuration file for our guest is at \texttt{/etc/xen/lab14img.cfg}. Review it.
 \item An operating system has been installed on our volume. we will see this in the configuration
          and can verify it further by running the guest. Start it now.
\end{enumerate}       
   
A log of the guest image creating process is written to the \texttt{/var/log/xen-tools} directory. Review it now.

\section{Xen-Tools settings}
We know that image creation is a fairly complex policy with a large number of options. \texttt{xen-create-image} 
gets some of those options from the command line with which it is invoked. Things that aren't specified there 
come from the configuration file at \texttt{/etc/xen-tools/xen-tools.conf}. Check that file to see what the defaults
are. For example, how much disk space is dedicated to a guest by default?

Earlier we noted that Xen-tools could install a number of possible Linux distributions. To see which ones we can install on our system, check the directory \texttt{/usr/share/xen-tools}. There we will find installation resources for various 
distributions.

Run \texttt{xen-create-image} a second time, this time with one of the Debian distribution options. See 
\url{http://manpages.ubuntu.com/manpages/trusty/man8/xen-create-image.8.html} for information on 
options and usage.

\section{Other tools}
Besides \texttt{xen-create-image}, Xen tools includes some other scripts:

\begin{itemize}
  \item \texttt{xen-update-image}
  \item \texttt{xen-list-images}
  \item \texttt{xen-delete-image}
 \end{itemize}
 
 All of these are also documented at the same web site where \texttt{xen-create-image} is. For example, 
 \texttt{xen-update-image} will install OS updates on Debian-ish images that are not active. Try it now by 
 making sure that the guest we created earlier is not running, then run
 
 \texttt{xen-update-image --lvm=<hostname>-vg lab14img}
 
 again, substituting the correct value for \texttt{<hostname>}. To verify that it worked, start the guest and see 
 that there are no updates that need to be installed.
 
 \section{Conclusion}
 Xen-tools only addresses on small part of the Xen management problem, and only in a fairly limited way. It does, however, show an approach that can be applied more thoroughly to provide reasonably capable Xen management tools.
 
 











\end{document}