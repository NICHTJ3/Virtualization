\documentclass{article}
\usepackage{graphicx}
\usepackage{wrapfig}
%\usepackage{inconsolata}
\usepackage{enumerate}
\usepackage{hyperref}
\usepackage{verbatim}
\usepackage[parfill]{parskip}
\usepackage[margin = 2.5cm]{geometry}

\usepackage[T1]{fontenc}


\begin{document}

\title{Lab 11.1: Docker Volumes\\ IN720 Virtualisation}
\date{}
\maketitle

\section*{Introduction}
In this lab we are going to play around with Docker volumes and bind mounts to see how we can create and seed volumes with data that we can then share across other containers.


\section{Build a data container}
Build a new container from a Dockerfile that with the following:

\begin{enumerate}
	\item Create a directory, \texttt{/data} inside your container.
	\item In \texttt{/data}, create files named \texttt{foo}, \texttt{bar}, and \texttt{baz}.
	\item Make \texttt{/data} a volume.
\end{enumerate}

Build your image and launch a container called \texttt{vol\_lab\_1} from it.


\section{Find your volume on the host system}
In another ssh session, inspect your new container to find its volumes with the command

\texttt{sudo docker inspect -f \{\{.Mounts\}\} vol\_lab\_1}

Once you find the directory on the host that stores your volume, modify one of the files inside it.  Attach a terminal to \texttt{vol\_lab\_1} and see if the changes on the host system are visible inside the container.

\section{Using the volume in another container}
Shut down \texttt{vol\_lab\_1}.  Start a new container with the command

\begin{verbatim}
sudo docker run -it  --volumes-from=vol_lab_1 ubuntu /bin/bash
\end{verbatim}

When your new container starts, inspect the \texttt{/data} volume inside it.  Make some changes inside the volume and then shut down this container.  Then, check the volume directory on the host system and see if your changes are visible there.

\section{Creating a bind mount from an exisiting directory on the host}
Often, especially when developing applications, we have a directory on the host system that contains code or other resources we want to use inside a container while still keeping access to them directly on the host. Modify the container image you created earlier by doing the following:

\begin{enumerate}
	\item Create a directory, \texttt{/home/user/mydata} on your host system. Place some files in it.
	\item In your Dockerfile, remove the earlier commands that created the files in \texttt{/data}, but do create the \texttt{/data} directory.
	\item Rebuild your image and start a new container from it. Create a bind mount of \texttt{/home/user/mydata} onto \texttt{/data}
	\item On your host system, modify the contents of \texttt{/home/user/mydata} and then observe how those changes are visible inside your running container.
\end{enumerate}

\section{One final note}
The important property of volumes is that they exist independently of the containers that use them. You can see what volumes are present on your system with the command

\texttt{docker volume ls}

\end{document}