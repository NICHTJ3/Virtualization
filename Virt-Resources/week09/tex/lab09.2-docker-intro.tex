\documentclass{article}
\usepackage{graphicx}
\usepackage{wrapfig}
%\usepackage{inconsolata}
\usepackage{enumerate}
\usepackage{hyperref}
\usepackage{verbatim}
\usepackage[parfill]{parskip}
\usepackage[margin = 2.5cm]{geometry}

\usepackage[T1]{fontenc}


\begin{document}

\title{Lab 9.2 Introduction to Docker\\ IN720 Virtualisation}
\date{}
\maketitle

\section*{Introduction}
Last time we saw how we could use LXC to create and run containers on a host. While LXC is powerful and useful, its interface is a little low level. \emph{Docker} is a tool that builds on the ideas we saw in LXC and extends our abilities to create and manage containers.  In this lab we will start working with a more full featured and widely used container toolset \emph{Docker}.

\section{Setup}
Use the same server that you used for the LXC lab.  Install the Docker tools with the command

\texttt{sudo apt-get install docker.io}

Right now, the docker commands are only usable by \texttt{root}. The Docker package created a \texttt{docker} group, however, and any member of that groupc can run commands without \texttt{sudo}. Add your user to the \texttt{docker} group with the command \texttt{sudo adduser user docker}.

After installing, check to see that your Docker server is running with the command 

\texttt{docker info}

You should see output like the following:

\begin{verbatim}
content..Containers: 0
Images: 0
Storage Driver: aufs
Root Dir: /var/lib/docker/aufs
Dirs: 6
Execution Driver: native-0.2
Kernel Version: 3.13.0-59-generic
WARNING: No swap limit support
.
\end{verbatim}

\section{Creating and running a container}

Create a new container with the command

\begin{verbatim}
docker run -i -t --name fred ubuntu /bin/bash
\end{verbatim}

This will create a new container named \texttt{fred} based on the ubuntu base image. We have told docker to run \texttt{bash} on the container, and the \texttt{-i} and \texttt{-t} options connect us to an interactive console on it.  

Once the container is up you can interact with it normally.  A few things about your container environment are interesting. Run \texttt{top} to see what is running inside the container.  For comparison, you mnay want to run \texttt{top} on the host system when you exit the container.  Also, use \texttt{ip a} to inspect the container's network interfaces.

Type \texttt{exit} to return to the host. Since this terminates the \texttt{bash shell} the container itself stops.

On the host system, type \texttt{docker ps -a} to list the containers on the system. (Without the \texttt{-a} it will only show running containers) You can get more information about your container with the command

\texttt{docker inspect fred}

Now restart your container with the command 

\texttt{docker start fred}

and run \texttt{sudo docker ps}.  You will see that the container is running, but we are not attached to the console.  You can attach to is with the command

\texttt{docker attach fred}

\section{Next steps}
We don't usually want to create containers that we have to deal with interactively.  To see how to create and use \emph{daemonized containers}, work through  Chapter 3, sections 7-17 of the \emph{The Docker Book}.




\end{document}
