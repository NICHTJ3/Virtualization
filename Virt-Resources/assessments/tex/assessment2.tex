\documentclass{article}
\usepackage{graphicx}
\usepackage{wrapfig}
%\usepackage{inconsolata}
\usepackage{enumerate}
\usepackage{hyperref}
\usepackage{verbatim}
\usepackage[parfill]{parskip}
\usepackage[margin = 2cm]{geometry}

\usepackage[T1]{fontenc}


\begin{document}

\title{Assignment 2: OpenStack \\ IN720 Virtualisation}
\date{}
\maketitle

\section*{Introduction}
One of the themes of this paper is that \emph{infrastructure is code}. To prove our point we will write some code to create and control elements of infrastructure in OpenStack. You will use the OpenStack Python SDK to write a script to do this.

This assignment is worth 20\% of your mark in this paper.

\section{Specifications}
See the GitHub repository at \url{https://github.com/tclark/virt-openstack-assignment}. The instructions are in the README. You will need to fork this repository into a repository of your own.

You are free to discuss your work with other students and consult any references you wish, but the code you submit must be your own. If necessary you should be able to explain exactly how your code works. It is understood that the degree of Python experience of students in the class varies widely, feel free to ask questions and seek help if you need it.

\section{Submission}
You assignment will be evaluated on its functionality and the overall quality of the code. Submit your work by sending a pull request to the original source repository.

This assignment is due on Monday, 14 October at noon.  

\end{document}